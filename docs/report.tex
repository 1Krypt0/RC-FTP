\documentclass[11pt]{article}

\usepackage{graphicx}
\usepackage[utf8]{inputenc}
\usepackage[margin=3.5cm]{geometry}
\graphicspath{{./images/}}

\title{\includegraphics[scale=0.3]{logo.png} \\ \textbf{File Transfer Protocol}}
\author{Computer Networks\\ Bachelors in Informatics and Computing Engineering \\ \\ 3LEIC03\_G6 \\ \\ Tiago
Rodrigues up201907021@fe.up.pt \\ Mário Travassos up201905871@fe.up.pt  }
\date{\today}

\begin{document}

\maketitle

\newpage

\section*{Summary}

\paragraph{}This report will cover the second project proposed for the Computer Networks Curricular Unit, which had the objective of developing an application that would download files using the FTP standard, along with setting up a network between the labs computers.

\paragraph{}The application supports both anonymous and authenticated downloads, and the arguments must be valid URL's, without the port.

\section*{Introduction}

\paragraph{}The second project had two main objectives: The development of the download application, and the configuration of the computer network. This report will examine both sections, detailing the architecture of the application and the setup used to launch the network. The application can be used to download files from any FTP server, since it adopts the standard URL syntax.

\paragraph{}The report is structured as follows: First, the application is described in detail, breaking down the architecture and the more relevant components. Then, a report of a successful download is made. After that, an analysis of the network will take place. Finally, a set of relevant attachments will be included.

\section*{Download Application}
\subsection*{Architecture}
\subsection*{Case study}
\section*{Network Configuration and Analysis}
\section*{Conclusions}
\section*{References}
\section*{Annexes}
\subsection*{Application Source Code}
\subsection*{Configuration Commands}
\subsection*{Data Logs}

\end{document}
